\documentclass{MDXHandbook}
\usepackage{todonotes}
\hypersetup{
    colorlinks,
%    linkcolor={red!50!black},
%    citecolor={blue!50!black},
%    urlcolor={blue!80!black}
    linkcolor={MDXCorporateRed},
    citecolor={blue},
    urlcolor={blue}
}

%%% NOTE: Please compile this file at least twice as the title page uses exact positioning that need two compilations

%%%%%%%%%%% Please include details of the module
\modulecode{[MODULE CODE]}
\moduletitle{[MODULE NAME]}
\moduleleader{A.N.Other} 
\term{AY (2019)}
\duration{24 weeks}
\version{1}
\dept{Design Engineering and Mathematics}
%%%%%%%%%%% This is used by MDXHandbook.cls to format the title page

\begin{document}
\maketitle

\section*{Online location of handbook}
\todo[inline]{delete if module handbook is only available online}
This handbook can also be accessed via MyLearning.

\section*{Other formats available}
This handbook is available in a large print format. If you would like a large print copy or have other requirements for the handbook, please contact the Disability Support Service (\href{mailto:disability@mdx.ac.uk}{disability@mdx.ac.uk}.

\section*{Disclaimer}
The material in this handbook is as accurate as possible at the date of production. You will be notified of any minor changes. If there are any major changes to the module you will be consulted prior to the changes being confirmed. Please check the version number on the front page of this handbook to ensure that you are using the most accurate information.

\section*{Other documents}
Your module handbook should be read and used alongside your programme handbook and the information available to all students on My Learning, including the Academic Regulations. Your programme handbook can be found on the My Learning programme page.

\newpage
\tableofcontents

\newpage
\section{Welcome -- Module Introduction}
%%% Please amend as needed
\todo[inline]{Write a brief introduction to the course. Indicate any module specific requirements and indicate what is expected from students.}

\section{The Module Team}
%%% Please amend as needed
\todo[inline]{Please provide details of key staff. 
It may not be possible to include details of all staff for modules with large teaching teams and so please include details of key staff and indicate who to contact.
If photos are available you may wish to include them. 
Role – Module leader, lecturer, seminar/lab/workshop tutor, technical tutor, GAA etc
Room number – remove this if in open plan offices or hourly paid academic
You may wish to include office hours if information is available or include details of where to locate this information online.
 Include subject librarian and GAAs/SLAs attached to the module if applicable

For example:}
%%% \moduleStaff{Staff name} is a macro that produces a table for the member of staff
%%% Keys:
%%%		role = eg Module Leader, GAA, tutor, lab coordinator etc
%%% 		room = room number
%%%		telephone, tel = telephone number (up to you which format)
%%%		email = email address (n.b. @ is a perfectly fine in LaTeX (no need to use \@)
%%% 		photo = the filename or path name of an image file including its extension (NB there is a bug which means that filenames can't have spaces) 
\moduleStaff[role = Module Leader, room=TG50A, telephone = 020 84115555, email = \href{mailto:a.n.other@mdx.ac.uk}{a.n.other@mdx.ac.uk}, photo=NoPhoto.pdf]{A.N. Other}

\moduleStaff[photo = NoPhoto.pdf, role = Tutor, tel = x5555]{D.Alighieri}

\newpage
\section{Communication with the teaching team}
%%% Please amend as needed
\todo[inline]{In this section detail the ways staff will communicate with students and how students can communicate with staff.
For example}

Students may contact staff via e-mail, phone, by dropping in to staff office hours, and by making an appointment to see them outside office hours. 
Staff will contact students by e-mail, phone, the My Learning module page and via lectures and seminars.

The team may send urgent group and/or individual messages about the module to you by email, so it is important that you read your University email regularly.
All staff have office hours, it is not necessary to book an appointment during these hours, you just need to drop-in. 

In the first instance problems should be dealt with by talking to a member of the module team. You can give feedback on this module to the module leader, your Student Voice Leader, to your personal tutor, and through the end of module evaluation survey.

Our most important consideration is your health, wellbeing and safety as well as our staff and people related to the University. Remember that you -- as part of \#TeamMDX -- can stay up-to-date with the guidance on Coronavirus at \url{https://unihub.mdx.ac.uk/coronavirus-covid19}.


\section{Module Overview}
%%% Please amend as needed
\todo[inline]{(Please insert the Module Narrative here, as most recently approved, for 2021-22 delivery)

To obtain your APPROVED latest narrative please contact your programme administrator detailed below.
}
\begin{itemize}
	\item Natural Sciences: Stephanie Bee \href{mailto:S.Bee@mdx.ac.uk}{S.Bee@mdx.ac.uk}, Vanessa Pillay \href{mailto:V.Pillay@mdx.ac.uk}{V.Pillay@mdx.ac.uk} 
	\item Design Engineering Mathematics: Andrew Roberts \href{mailto:A.B.Roberts@mdx.ac.uk}{A.B.Roberts@mdx.ac.uk}, Susy Ryan \href{mailto:S.Ryan@mdx.ac.uk}{S.Ryan@mdx.ac.uk} 
	\item Computer Science: Andrew Roberts \href{mailto:A.B.Roberts@mdx.ac.uk}{A.B.Roberts@mdx.ac.uk}, Susy Ryan \href{mailto:S.Ryan@mdx.ac.uk}{S.Ryan@mdx.ac.uk}
	\item London Sport Institute: Monica Izarra Millan \href{mailto:M.IzarraMillan@mdx.ac.uk}{M.IzarraMillan@mdx.ac.uk}, Colin Allison \href{mailto:C.Allison@mdx.ac.uk}{C.Allison@mdx.ac.uk}, Sonia Dubois \href{mailto:S.Dubois@mdx.ac.uk}{S.Dubois@mdx.ac.uk}, Helen Michael \href{mailto:H.Michael@mdx.ac.uk}{H.Michael@mdx.ac.uk} 
	\item Psychology: Monica Izarra Millan \href{mailto:M.IzarraMillan@mdx.ac.uk}{M.IzarraMillan@mdx.ac.uk}, Colin Allison \href{mailto:C.Allison@mdx.ac.uk}{C.Allison@mdx.ac.uk}, Sonia Dubois \href{mailto:S.Dubois@mdx.ac.uk}{S.Dubois@mdx.ac.uk}, Helen Michael \href{mailto:H.Michael@mdx.ac.uk}{H.Michael@mdx.ac.uk}
\end{itemize}
Include:
\begin{itemize}
\item \emph{Aims}
\item \emph{Learning outcomes}
\item \emph{Syllabus}
\item \emph{Learning and teaching strategy}
\item \emph{Assessment scheme}
\end{itemize}

\emph{Also provide detail of}
\begin{itemize}
\item \emph{Learning hours}
\item \emph{Scheduled teaching  e.g. 30\%}
\item \emph{Independent study e.g. 70\%}
\item \emph{Placement (if applicable)}
\end{itemize}

\section*{Research Ethics}
\todo[inline]{Please insert the relevant statement , if appropriate, (select from one of three statements)}
\subsection*{Statement 1:}
\begin{itemize}
	\item This module will require you to undertake an ethical review process/application before you carry out any research involving human participants, human data, animals/animal products, precious artefacts, materials or data systems.
	\item Data MUST not be collected without first obtaining ethics approval for your research. If you submit a project that includes data gathered from or about people without ethical approval this may be treated as academic misconduct and could lead to fail grade being awarded.
	\item Research ethics approval seeks to ensure all work is designed and undertaken according to certain principles of ethical research. These include: 
		\begin{enumerate}
			\item Primary concern must be given to the safety, welfare and dignity of participants, researchers, colleagues, the environment and the wider community 
			\item Consideration of risks should be undertaken before research commences with the aim of minimising risks to those involved – i.e. human participants or animal subjects, colleagues, the environment and the wider community, as well as actual or potential risks to those directly or indirectly affected by the research.
			\item Informed consent should be freely given by participants, and by a trained person when collecting or analysing human tissue (details on accessing and completing online training for gaining informed consent for HTA purposes can be found below in Section 8).
			\item Respect for the privacy, confidentiality and anonymity of participants 
			\item Consideration of the rights of people who may be vulnerable (by virtue of perceived or actual differences in their social status, ethnic origin, gender, mental capacities, or other such characteristics) who may be less competent or able to refuse to give consent to participate
			\item Researchers have a responsibility to the general public and to their profession; as such they should balance the anticipated benefits of their research against potential harm, misuse or abuse which must be avoided 
			\item Researchers must demonstrate the highest standards of ethical conduct and research integrity. They must work within the limits of their skills, training and experience, and refrain from exploitation, dishonesty, plagiarism, infringement of intellectual property rights and the fabrication of research results. They should declare any actual or potential conflicts of interest, and where necessary take steps to resolve them. 
			\item When using human tissues for research, Human Tissue Act and Human Tissue Authority (HTA) requirements must be met. Please contact the relevant designated person (DP) in your department or the HTA Designated Individual (DI) (Dr Lucy Ghali - \href{mailto:L.Ghali@mdx.ac.uk}{L.Ghali@mdx.ac.uk}). Further information is provided below in the section: ``Human Tissue Authority Information'', see ``Governance Structure'' document and SOPs etc.
			\item Research should not involve any illegal activity, and researchers must comply with all relevant laws
		\end{enumerate}
	\item You can apply for research ethical approval using the Middlesex Online Research Ethics (MORE) system which has information and guidance to help you meet the highest standards of ethical research using this link: \url{MOREform.mdx.ac.uk}
	\item Information and further guidance on how to complete a research ethics application form (e.g., video guides and templates) can be found on the MORE MyLearning site\textsuperscript{$\ast$}: \url{mdx.mrooms.net/enrol/index.php?id=12277} (Log in required)\\ 
 \textsuperscript{$\ast$}Middlesex University Definition of Research document can be located on this site.
\end{itemize}

\subsection*{Statement 2:}
\begin{itemize}
	\item 	This module has a set of pre-approved of ethical protocol’s which you are required to follow as part of carrying out any research involving human participants, human data, animals/animal products, precious artefacts, materials or data systems.
	\item If you submit a project that includes data gathered from or about people outside of the ethical protocols approved this may be treated as academic misconduct and could lead to fail grade being awarded. This means only collecting data types and using data gathering methods as outlined by your module tutor.
	\item Research ethics approval seeks to ensure all work is designed and undertaken according to certain principles of ethical research. These include: 
	\begin{enumerate}
		\item Primary concern must be given to the safety, welfare and dignity of participants, researchers, colleagues, the environment and the wider community 
		\item Consideration of risks should be undertaken before research commences with the aim of minimising risks to those involved – i.e. human participants or animal subjects, colleagues, the environment and the wider community, as well as actual or potential risks to those directly or indirectly affected by the research.
		\item Informed consent should be freely given by participants, and by a trained person when collecting or analysing human tissue (details on accessing and completing online training for gaining informed consent for HTA purposes can be found below in Section 8).
		\item Respect for the privacy, confidentiality and anonymity of participants 
		\item Consideration of the rights of people who may be vulnerable (by virtue of perceived or actual differences in their social status, ethnic origin, gender, mental capacities, or other such characteristics) who may be less competent or able to refuse to give consent to participate
		\item Researchers have a responsibility to the general public and to their profession; as such they should balance the anticipated benefits of their research against potential harm, misuse or abuse which must be avoided 
		\item Researchers must demonstrate the highest standards of ethical conduct and research integrity. They must work within the limits of their skills, training and experience, and refrain from exploitation, dishonesty, plagiarism, infringement of intellectual property rights and the fabrication of research results. They should declare any actual or potential conflicts of interest, and where necessary take steps to resolve them. 
		\item When using human tissues for research, Human Tissue Act and Human Tissue Authority (HTA) requirements must be met. Please contact the relevant designated person (DP) in your department or the HTA Designated Individual (DI) (Dr Lucy Ghali - \href{mailto:L.Ghali@mdx.ac.uk}{L.Ghali@mdx.ac.uk}). Further information is provided below in the section: ``Human Tissue Authority Information'', see ``Governance Structure'' document and SOPs etc.
		\item Research should not involve any illegal activity, and researchers must comply with all relevant laws
	\end{enumerate}
	\item For more information about ethics go to the Middlesex Online Research Ethics (MORE) system which has information and guidance to help you meet the highest standards of ethical research using this link: \url{MOREform.mdx.ac.uk}
	\item Information and further guidance on how to complete a research ethics application form (e.g., video guides and templates) can be found on the MORE MyLearning site\textsuperscript{$\ast$}: \url{mdx.mrooms.net/enrol/index.php?id=12277} (Log in required)\\
\textsuperscript{$\ast$}Middlesex University Definition of Research document can be located on this site.
\end{itemize}

\subsection*{Statement 3:}
	\begin{itemize}
		\item The teaching, learning, assessment and research activities undertaken in this module have been considered and are not likely to require ethical approval. 
		\item However, please seek advice if undertaking the module entails carrying out any research activities involving human participants, human data, animals/animal products, precious artefacts, materials or data systems. If you submit work that includes data gathered from or about people, this may be treated as academic misconduct and could lead to fail grade being awarded. 
		\item Research ethics approval seeks to ensure all research is designed and undertaken according to certain principles of ethical research. These include: 
		\begin{enumerate}
			\item Primary concern must be given to the safety, welfare and dignity of participants, researchers, colleagues, the environment and the wider community 
			\item Consideration of risks should be undertaken before research commences with the aim of minimising risks to those involved – i.e. human participants or animal subjects, colleagues, the environment and the wider community, as well as actual or potential risks to those directly or indirectly affected by the research.
			\item Informed consent should be freely given by participants, and by a trained person when collecting or analysing human tissue (details on accessing and completing online training for gaining informed consent for HTA purposes can be found below in Section 8).
			\item Respect for the privacy, confidentiality and anonymity of participants 
			\item Consideration of the rights of people who may be vulnerable (by virtue of perceived or actual differences in their social status, ethnic origin, gender, mental capacities, or other such characteristics) who may be less competent or able to refuse to give consent to participate
			\item Researchers have a responsibility to the general public and to their profession; as such they should balance the anticipated benefits of their research against potential harm, misuse or abuse which must be avoided 
			\item Researchers must demonstrate the highest standards of ethical conduct and research integrity. They must work within the limits of their skills, training and experience, and refrain from exploitation, dishonesty, plagiarism, infringement of intellectual property rights and the fabrication of research results. They should declare any actual or potential conflicts of interest, and where necessary take steps to resolve them. 
			\item When using human tissues for research, Human Tissue Act and Human Tissue Authority (HTA) requirements must be met. Please contact the relevant designated person (DP) in your department or the HTA Designated Individual (DI) (Dr Lucy Ghali - \href{mailto:L.Ghali@mdx.ac.uk}{L.Ghali@mdx.ac.uk}). Further information is provided below in the section: ``Human Tissue Authority Information'', see ``Governance Structure'' document and SOPs etc.
			\item Research should not involve any illegal activity, and researchers must comply with all relevant laws.
		\end{enumerate}
	\item For more information about ethics go to the Middlesex Online Research Ethics (MORE) system which has information and guidance to help you meet the highest standards of ethical research using this link: \url{MOREform.mdx.ac.uk}
	\item Information and further guidance on how to complete a research ethics application form (e.g., video guides and templates) can be found on the MORE MyLearning site\textsuperscript{$\ast$}: \url{mdx.mrooms.net/enrol/index.php?id=12277} (Log in required)\\ 
\textsuperscript{$\ast$}Middlesex University Definition of Research document can be located on this site.
\end{itemize}

\section{Learning Resources}
%%% Please amend as needed
\todo[inline]{When your reading list is available online please include a link here to your list at \url{readinglists.mdx.ac.uk} rather than listing materials here. For more information please contact your Liaison Librarian: \url{libguides.mdx.ac.uk/liaisonlibrarians}}

\todo[inline]{Please insert hyperlinks to the online resources or any other relevant materials such as podcasts, e-books, videos.}

\todo[inline]{Please include some information on the variety of resources used to support learning for the module. For example}

This module has a variety of learning resources available for you to use to support your learning. These include module notes, worked examples, solutions to exercises, feedback, podcasts, and key reading materials. These can be accessed online via the module page. Please visit the module page regularly to make use of these. 


\section{Making the most of this module}
%%% Please amend as needed
\todo[inline]{Please indicate here how students should engage with the module and any module specific requirements. For example}

The module team are here to help and support you achieve your goals. One of the key elements to successfully completing this module is engaging with all of the learning opportunities we offer as well and working with your peers to support one another.  

\todo[inline]{Detail what students are required to do to engage in the module specify any requirements. For example}

This module is designed as a combination of contact sessions, directed study and independent study. This means you must participate in all the allocated sessions and you must complete all set prework and activities outside them. Students are expected to take an active part in all learning sessions whether these are online or on campus; lectures, lab sessions, practical classes, seminars and workshops.  

To make the most of this module please complete the following every week 

\begin{itemize}
	\item Complete all prework in preparation for learning sessions. This may be watching videos, reading through set material or chapters and completing activities. Please make notes of points you need to clarify and discuss these in learning sessions with module tutors. 
	\item Read through the notes making a note of any points you need to discuss with your tutor. 
	\item Complete the set activities before the next session, making a note of any points you need to discuss with your tutor. 
	\item Go to the module My Learning page, attempt the quizzes, make use of extra material, view the podcasts, and access the activity solutions. Make a note of anything you wish to discuss with your tutor. 
	\item Complete further reading from the core text online. 
\end{itemize}
The module team is committed to support you and your fellow students whilst you undertake this module. In order for you to get the most out of sessions you need to come prepared and ready to contribute. Please ensure that any work set by the team has been completed before workshops. After each class please review what has been covered and make a note of anything you would like clarification on.  

Engaging with online and on-campus in-person learning and activities is integral to your success.  Middlesex University supports students, enabling them to achieve their full potential.  

We provide this support through a number of strategies, all of which provide our students with a supportive learning environment online, remotely, face-to-face, or blended. 

Further information on engaging with your programme will be available at your Induction and updates online at UniHub 
\url{https://unihub.mdx.ac.uk/study/assessment/attendance}

\subsubsection*{Professional behaviour and online conduct}
The programme of study you are undertaking is underpinned by developing professional behaviour and attitude.  

It is important that you are respectful and supportive to your fellow students and tutors. Adopting this approach will create a positive atmosphere within sessions and is something you can use in your professional life.  

You must come to sessions prepared and ready to contribute where appropriate.   

To access some of the rooms and specialist space used for this module you will need your University ID card. Please remember that your University ID should be carried with you always. 

Please conduct your email communication with fellow students, tutors and all relevant staff in a courteous manner. It is helpful to provide your student number and if you have a query relating to a module include the module number as well.  

In the same way that we help you understand how to effectively participate in learning on campus, we also want to make sure that you can make the most of online learning. Our principles of online learning class conduct are available at:
\url{https://unihub.mdx.ac.uk/covid-19-updates-faq/online-classroom-conduct}

\subsubsection*{Academic Integrity and Misconduct}
You should be aware of the University’s academic integrity and misconduct policies and procedures. Taking unfair advantage over other students in assessment is considered a serious offence by the University. Action will be taken against any student who contravenes the regulations through negligence, foolishness or deliberate intent. Academic misconduct takes several forms, in particular:  
\begin{itemize}
	\item Plagiarism --- using extensive unacknowledged quotations from, or direct copying of, another person’s work and presenting it for assessment as if it were your own effort. This includes the use of third party essay writing services. 
	\item Collusion --- working together with other students (without the tutor’s permission), and presenting similar or identical work for assessment. 
	\item Infringement of Exam Room Rules --- Communication with another candidate, taking notes to your table in the exam room and/or referring to notes during the examination. 
	\item Self-Plagiarism --- including any material which is identical or substantially similar to material that has already been submitted by you for another assessment in the University or elsewhere.
\end{itemize}
Students who attempt to gain unfair advantage over others through academic misconduct  will be penalised by sanctions, according to the severity of the offence, which can include exclusion from the University. Links to the relevant University regulations and additional support resources can be found here:

Full details on academic integrity and misconduct and the support available can be found at  Academic Integrity | UniHub (\url{mdx.ac.uk}) 

\textbf{Becoming a successful student} Course which includes Academic Integrity

\textbf{Access to course.} You will have to log into to MyUniHub and then MyLearning to access the course.

The Academic Integrity and Misconduct policy is available in our Public Policy Statements (under Academic Quality) at: Our policies | Middlesex University London (\url{mdx.ac.uk})

Referencing \& Plagiarism: Suspected of plagiarism?:
\url{http://libguides.mdx.ac.uk/c.php?g=322119&p=2155601}

Referencing and avoiding plagiarism:
\url{https://unihub.mdx.ac.uk/study/writing-numeracy/awl-resources/writing}

The Middlesex University Students’ Union (MDXSU) Advice Service offers free and independent support  in making an appeal, complaint or responding to any allegations of academic or non-academic misconduct. 
\url{https://www.mdxsu.com/advice}

\subsubsection*{Extenuating circumstances}
There may be difficult circumstances in your life that affect your ability to meet an assessment deadline or affect your performance in an assessment. These are known as extenuating circumstances or ‘ECs’. Extenuating circumstances are exceptional, seriously adverse and outside of your control. Please see link for further information and guidelines:
\url{https://unihub.mdx.ac.uk/your-study/assessment-and-regulations/extenuating-circumstances}

\section{Assessment}
%%% Please amend as needed
\textbf{Formative assessment}: Formative assessment is completed during your year of study and provides the opportunity to evaluate your progress with your learning.  Classroom assessment is one of the most common formative assessment techniques although other activities and tasks may be used. Formative assessments help show you and us that you are learning and understanding the material covered in this course and allow us to monitor your progress towards achieving the learning outcomes for module. Although formative assessments do not directly contribute to the overall module mark they do provide an important opportunity to receive feedback on your learning.
\todo[inline]{Please state any formative activities and deadlines for feedback on these within the module.}

\begin{center}
	\begin{tabular}{|p{.5\textwidth}|p{.4\textwidth}|}
		\rowcolor{MDXCorporateRed}\hline
		\multicolumn{1}{|c|}{\textcolor{white}{Formative assessment}} & \multicolumn{1}{|c|}{\textcolor{white}{Deadline}}\\
		\hline
		Add details & Date and time\\
		\hline
	\end{tabular}
\end{center}

\textbf{Summative assessment}: Summative assessment is used to check the level of learning at the end of the course. It is summative because it is based on accumulated learning during the course. The point is to ensure that students have met the learning outcomes for the course and are at the appropriate level. It is the summative assessment that determines the grade that you are awarded for the module.

There are \todo[inline]{Insert number} assessment components in this module \todo[inline]{List them.:

\emph{Include a table of deadlines for the module and then provide details for each in turn. Examples of how this can be done have been provided in a separate document.} For example:
}

\todo[inline]{(insert title e.g. Programming assignment)}

\todo[inline]{Provide the following details for each assessment }
\begin{itemize}
	\item \emph{nature of the assessment,}
	\item \emph{assessment brief}
	\item \emph{\% weighting, }
	\item \emph{deadline} 
	\item \emph{Type of feedback you will receive (written, oral, electronic, etc etc)}
	\item \emph{feedback return date. }
	\item \emph{marking criteria rubric. }
	\item \emph{additional grading criteria for example:}
	
	\emph{In order to pass this module, you need to achieve an aggregated pass overall and must achieve a minimum of grade 18 or above in all assessment tasks. 
	All threshold SOBs must be demonstrated to pass the module.}

	\item \emph{Please include any PRSB requirements, e.g. compensation not permitted. }
\end{itemize}

The table below specifies the associated deadlines:

\begin{center}
	\begin{tabular}{|p{.3\textwidth}|p{.15\textwidth}|p{.25\textwidth}|p{.25\textwidth}|}
		\rowcolor{MDXCorporateRed}\hline
		\centering\textcolor{white}{Summative assessment} &
		\centering\textcolor{white}{Weighting} &
		\centering\textcolor{white}{Deadline} &
		\multicolumn{1}{|p{.25\textwidth}|}{\centering\textcolor{white}{Feedback}} \\
		\hline
		add details & \% & date and time & return date\\
		\hline
		add details & \% & date and time & return date\\
		\hline
		add details & \% & date and time & return date\\
		\hline
	\end{tabular}
\end{center}

\section*{Overall module grade}
%%% Please amend as needed
\todo[inline]{Indicate how the overall module grade will be calculated. For example }

Each component of assessment will be marked directly onto the 20-point scale based on the assessment criteria. To produce the overall module grade a weighted average percentage will be calculated using the midpoint percentage (see section 9) and then converted to a 20-point grade using the University scale in the appendix. 

\todo[inline]{Or} 

Each component of assessment will be marked as a percentage. To produce the overall module grade a weighted average percentage will be calculated and then converted to the 20-point grade using the University scale in the appendix.  

In order to pass this module, you need to pass all assessment tasks with a minimum grade of X or equivalent.\todo[inline]{(Please include any Professional, Regulatory or Statutory Body (PRSB) requirements)}

Before you submit your work for final grading, please ensure that you have accurately referenced the work.  It is your responsibility to check the spelling and grammar.  If you have submitted a formative or draft assessment, you will receive feedback but no grade. The comments should inform you about how well you have done or tell you about the areas for improvement. All assignments should be submitted online unless specified in assessment briefs.

Reassessment for this module normally takes place in the following way:
\todo[inline]{add detail on when reassessment is normally held, include any PSRB requirements that constrain reassessment; where the module is by continuous assessment or reassessment eg SOBs indicate that this provides the reassessment opportunity; where students have the opportunity to undertake reassessment in year indicate that this provides the reassessment opportunity.}

Further information is available at \\ \url{https://unihub.mdx.ac.uk/study/assessment/regulations}

Middlesex University is committed to being fair in its approach to assessing student learning following the UK Quality Code for Higher Education (Quality Code) (2018) and the UK Quality Code - Advice and Guidance: Assessment (2018) and External Expertise (2018).

The Assessment Fairness guidance, policies and procedures put in place by Middlesex University in our commitment to ensure fairness for all in assessment, include our Academic Policy Statement APS18: Curriculum Design Policy (2018), Middlesex University Regulatory Framework Code of Assessment Practice: Section M, Academic Policy Statement APS29: Anonymous Marking Assessment Policy (2020), Equality and Diversity Policy and Codes of Practice (HRPS8), specifically code of practice 7: Curriculum, Pedagogy and Assessment and Key Principles of Assessment. 

If you have any queries or would like to know more on how this approach has been applied to modules you are studying please contact your Programme Leader.

\subsection{Feedback on your assignments}
You will be provided with feedback on all coursework that is helpful and informative, consistent with aiding the learning and development process. The nature of the feedback shall be determined at programme level but may take a variety of forms including: written comments; individual and group tutorial feedback; peer feedback; or other forms of effective and efficient feedback. 
Feedback will normally be provided within 15 WORKING DAYS of the published coursework component submission date.

\subsection{How is your assignment mark agreed?}
The following diagram provides an overview of the marking process for your module assessment. Further information on the role of  external examiners can be found at \url{https://www.mdx.ac.uk/about-us/policies/academic-quality/handbook} (section 4)

\begin{center}
\begin{tikzpicture}
\newcommand{\arrownode}[3]{
		\draw #1 ++ (2,0.75) node  [right, rectangle, draw, thick, minimum height = 1.5cm, text width = {.85\textwidth}] {$\cdot$ #2};
		\filldraw [thick, draw = black, fill = MDXCorporateRed] #1 --++ (1,-1) --++ (1,1) --++ (0,1.5) --++ (-1,-1) --++ (-1,1) -- #1;
		\draw #1 ++ (1,0.25) node [yshift = -4pt] {\textbf{\textcolor{white}{#3}}};
}
\foreach \i in {0,...,4} \coordinate (I\i) at (0,{-2 * \i});

\arrownode{(I0)}{You submit your assignment}{1}
\arrownode{(I1)}{The first marker grades the work and provides feedback; this could be completed anonymously depending on the assessment type.}{2}
\arrownode{(I2)}{A moderator or second marker reviews a sample of the work to quality assure the grades and feedback, to ensure they are accurate.  A final mark for the work is agreed between the first marker and the moderator or second marker.}{3}
\arrownode{(I3)}{A sample of work is sent to the External Examiner to check that the grading and feedback is at the right level and in line with external subject benchmarks (this applies to levels 5 \& 6 only)}{4}
\arrownode{(I4)}{Your final grades are submitted to the subject assessment board.}{5}
\end{tikzpicture}
\end{center}

\subsection*{Anonymous Marking Assessment Policy}
We have worked with the MDXSU to create an anonymous marking policy, in response to student feedback.  Anonymous marking ensures that your identity (your name, student number and other personal/identifiable information) is not made available to academics when they are marking your work.  This means that you can have confidence that your assessments will be marked fairly and consistently.  However, there are some forms of assessment for which anonymity cannot be guaranteed and these are recognised in the policy.  We believe that it is important to provide you with the support and guidance needed to help you develop and prepare for your final assessments (those which count towards your final grades i.e. summative assessments).  Therefore, anonymous marking will not apply to learning activities and assessments that do not contribute to your final grades (i.e. formative assessments).  If you require further information and support to understand how anonymous marking works in your programme modules please contact the Module Leader for more information.

The Anonymous Marking Assessment Policy is available at: 
\url{https://www.mdx.ac.uk/__data/assets/pdf_file/0037/563599/anonymous-marking-assessment-policy.pdf}

We now look at each component of assessment for this module in detail. Each of the following tables provides an overview of the requirements for each component. The support provided for each component along with the feedback arrangements, is also detailed below. \todo[inline]{(amend as appropriate)}


\newpage
\begin{landscape}
	\subsubsection{Assessment 1 (enter name of assessment)}
	The following table provides an example of the overview of the assessment requirements. 

	\begin{tabular}{|>{\bfseries}p{.35\linewidth}|p{.55\linewidth}|}
		\rowcolor{MDXCorporateRed}\hline 
		\multicolumn{2}{|c|}{\textcolor{white}{Example --- Assessment brief presentation}}\\
		\hline
		Module code & Insert the module code.\\ \hline
		Module title & Insert the module title.\\\hline
		Submission date, time & Clearly state the submission date, time and whether it is online or not.\\\hline
		Feedback type and date  & Please specify how (the type) / when (the date) students will receive feedback on their submitted work.\\\hline
		Word count & State the word count or equivalent for the assignment.\\\hline
		Assignment type & Please indicate the type of assignment e.g. presentation, essay, case study etc.\\\hline
		Assignment structure, format and details & Outline how you would like the assignment to be structured.

		Try to be explicit in terms of the formatting that is expected for the assignment. 

		This could be clearly stating the font style, number of slides etc. 
		Please state the type of referencing you want students to use. 

		Please also include any additional details that are necessary. 

		You can direct students to \url{http://www.citethemrightonline.com}.
		\\\hline
		Assessed learning outcome (s) & Please cut and paste from the module narrative.\\\hline
		Assessment weighting  & Insert a figure of the module weighting of the assignment.\\\hline
		Key reading and learning resources & Avoid generic statements like ‘you should utilise a wide range of sources’, rather offer students some guidance where appropriate on the key literature, theories and learning tools and resources they should be using in this assignment. \\
		\hline
	\end{tabular}

	\newpage
	\begin{tabular}{|p{.145\linewidth}|p{.145\linewidth}|p{.145\linewidth}|p{.145\linewidth}|p{.145\linewidth}|p{.145\linewidth}|}
		\hline
		\multicolumn{6}{|c|}{\textbf{Assessment marking criteria rubric (presentation)}} \\
		\hline
		\multicolumn{6}{|p{.87\linewidth}|}{
			\emph{As part of the assessment and fairness policy and guidance the university has approved a set of rubrics for use with all assignments. These are available for download and subject customisation from MyLearning and in Turnitin, for use with a structured general comment, see policy and guidance for more details}

			\url{https://www.intra.mdx.ac.uk/about-us/services/centre-for-academic-practice-enhancement/policy-bank/FV-Assessment-Fairness.pdf}
		}\\\hline

		Criteria &	1-4 First & 5-8 Upper Second & 9-12 Lower Second & 13-16 Third & 17-20 Refer \\\hline
		\textbf{Content:}
		
		Topic chosen relevant and informative, with a depth of knowledge and understanding.	
		
		& Excellent and well-informed understanding of theories and concepts involved with topic.
		& Good understanding of theories and concepts involved with topic.
		& Demonstrate satisfactory knowledge and understanding of topics theories and concepts.
		& Adequate content, and limited depth of knowledge and understanding.
		& Inadequate content and limited depth of knowledge and understanding.\\\hline

		\textbf{Communication:}
		
		Interesting, relevant language, explanation of terminology.
		
		& Very well expressed and very good understanding of content.
		& Very well expressed; good understanding of content.
		& Well expressed; understanding of content.
		& Unclear expression of information; little understanding of content.
		& Unclear and confusing; lack of understanding of content.
		\\\hline
		\textbf{References:}
		
		Reference to sources including directions for further study.
		& Broad and relevant readings examined and used selectively in presentation.
		& Good range of appropriate references used during the presentation.
		& Conventional references and readings used within presentation.
		& Adequate but limited use of references during presentation.
		& Critique relies on no or one reference; evidence of unexamined personal opinion.
		\\\hline
		\textbf{Technology:}
		Use of video, PowerPoint, tools etc. to present the chosen topic.
		& Very good selection, use and integration of technology.
		& Good use of appropriate AV.
		& Technology used but poor integration.
		& Technology used but poorly.
		& Absence of any technology use.
		\\\hline
		\textbf{Workload:}
		
		Balance between students.
		& All students share presentation role equally with good transition.
		& Presentation shared but with poor transition.
		& Presentation shared unequally.
		& Presentation shows little sharing.
		& No sharing, one student presents.
		\\\hline
	\end{tabular}


\newpage
The following table details the support you will be receiving for this assessment and the feedback opportunities you will have. 

\begin{tabular}{|p{.9\linewidth}|}
	\rowcolor{MDXCorporateRed}\hline
	\multicolumn{1}{|p{.9\linewidth}|}{\centering
	\textcolor{white}{\textbf{Support and draft feedback sessions for Insert assessment}}} \\\hline
	\textbf{Coursework briefing}

	Use this space to inform students of the date assessment briefing will take place. Please devote one session (preferably a seminar) to explaining assessment requirements. The session should specify the assessment criteria.
	\\\hline
	\textbf{Draft feedback opportunities}

	Please specify when/how students will receive feedback on their work in progress. Use this section to include information on when/if feedback on drafts will be given and how (in class, via email, during office hours etc). Also, this section to include information on extra feedback opportunities (e.g. clarification of any student questions via email, office hours, in class and/or a combination).

	It is advised that one seminar session is devoted to providing feedback/offering clarification to students. 
	\\\hline
	\textbf{Additional support}

	Please use this space to specify any additional support students can utilise. This could refer to:

	LET general workshops/feedback sessions
	
	LET specific sessions (in case you have arranged LET to be involved in your module)
	
	SLAs/GAAs – please specify what students should expect (e.g. help with citations, structure, explanation of jargon etc).
	\\\hline
\end{tabular}

\end{landscape}

\todo[inline]{Complete the details above for all of the coursework assessment components}

\newpage
\begin{landscape}
	\subsection{Exam}
	If your module has a final exam, a simplified version of both tables would be beneficial to our students. Please see below.
	
	The following table provides an example of the overview of the assessment requirements.
	
	\begin{tabular}{|>{\bfseries}p{.35\linewidth}|p{.57\linewidth}|}
		\rowcolor{MDXCorporateRed}\hline 
		\multicolumn{2}{|c|}{\textcolor{white}{Example --- Assessment brief presentation}}\\
		\hline
		Module code & Insert the module code.\\ \hline
		Module title & Insert the module title.\\\hline
		Exam date & Exam timetable will be published on \url{http://unihub.mdx.ac.uk}. \\\hline
		Feedback type and date  & Feedback on your exam script will be given after the module grades have been published. You will be informed by email of feedback opportunities.\\\hline
		Description of exam & Please provide a description of the type of exam, e.g. seen/unseen essay type questions/multiple choice etc., let students know the number of questions to be answered and the type of referencing required if applicable.\\\hline
		Duration of exam & Enter duration.\\\hline
		Assessed learning outcome (s) & Please cut and paste from the module narrative.\\\hline
		Module weighting  & Insert a figure of the module weighting of the assignment.\\\hline
		Key reading and learning resources & Avoid generic statements like ‘you should utilise a wide range of sources’, rather offer students some guidance where appropriate on the key literature, theories and learning tools and resources they should be using in this assignment. \\
		\hline
	\end{tabular}

	\begin{tabular}{|p{.35\linewidth}|p{.1\linewidth}|p{.1\linewidth}|p{.1\linewidth}|p{.1\linewidth}|p{.1\linewidth}|}
	\hline
	Marking criteria & 1-4 & 5-8 & 9-12 & 13-16 & 17-20 \\\hline
			& 70\%+ & 60\%-69\% & 50\%-59\% & 40\%-49\% & 39\% and below
		\\\hline	
	\end{tabular}
\end{landscape}

The following table details the support you will be receiving for the exam.

\begin{tabular}{|p{.9\linewidth}|}
	\rowcolor{MDXCorporateRed}\hline
	\multicolumn{1}{|p{.9\linewidth}|}{\centering
	\textcolor{white}{\textbf{Support and Feedback sessions for Exam}}}
	\\\hline
	\textbf{Exam briefing}

	Use this space to inform students of the week/date exam briefing will take place.
	\\\hline
	\textbf{Additional support}
	
	Please use this space to specify any additional support students can utilise. This could refer to:
	
	LET general workshops
	
	LET specific sessions, in case you have arranged for LET to be involved in your module
	
	SLA’s/GAA’s – please specify what students should expect, e.g. help with citations, structure, explanation of jargon etc.
	\\\hline
\end{tabular}

\begin{landscape}
\section{Learning Planner}
%%% Please amend as needed
\todo[inline]{Please distinguish between online and face-to face delivery
Please include a module schedule with details of what topics are covered when and any information relevant to the session. Give guidance in what students need to complete regularly.

This could be done week by week or by block of weeks. We recommend you indicate who will lead the sessions and give details of any work expected of the student before the session. Also highlight assessment and feedback points. 

Please note that the student activity section should cover all learning activities including online activities and reading along with other work set.

You may also wish to link the assessment and feedback to the modules learning outcomes to demonstrate when they have been achieved.

This could be done in the form of a table, example below. 
}
\newpage

\rowcolors{1}{MDXCorporateLightGrey!50}{white}
\begin{tabularx}{\linewidth}{|>{\centering}p{.075\linewidth}|>{\centering}p{.06\linewidth}|p{.2\linewidth}|p{.2\linewidth}|>{\centering}p{.04\linewidth}|>{\raggedright}X|>{\raggedright\arraybackslash}p{.15\linewidth}|}
\rowcolor{MDXCorporateRed}\hline %
\centering\color{white}Week beginning &	\centering\color{white}Week(s) &	\centering\color{white}Lecture & \centering\color{white}Workshop &	\centering\color{white}Staff & \centering\color{white}Student Activity & \centering\arraybackslash\color{white}Assessment and feedback\\ \hline
Sep 30th &	1	& Discrete probability: counting methods &	 Calculating probabilities. Discuss properties of distributions. Meet the SLAs & AM	& Review probability unit 1. Compete extension activities. Seek advice from tutors or SLAs.	 & Select presentation topic and notify module leader \\ 
\hline
Oct 7th & 2 & & & & & \\
\end{tabularx}

\todo[inline]{Rename lectures as appropriate. Give detailed descriptions of in-class activities for each day of the workshops.}

\end{landscape}

\newpage

		\todo[inline]{It should be made clear how each assessment component is marked whether this be in percentages or marked directly to the 20 point scale. It is recommended that it is made clear how the overall module grade is calculated from the component grades following the latest university guidance. It is also recommended that the REVISED University wide 20 point scale is included. Delete columns that do not apply.}

		%% \TwentyPointGradeTable produces the correct table for your subject
		%% Keys:
		%%		subject		=	MSO or PDE (others can be coded in the .cls file if necessary)
		%%		level			= 	UG or PG
		%%		width		= table width (default is \textheight assuming it is rotated using the sideways environment
		\TwentyPointGradeTable[subject = MSO, level = UG, width = .95\linewidth]
\end{document}